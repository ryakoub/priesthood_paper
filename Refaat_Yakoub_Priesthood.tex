\documentclass[12pt,doc]{apa7}   	% use "apa7" for two columns format
\usepackage[authoryear,round]{natbib}


\title{Priesthood's Roles in the Divinity Plan}	% Title of the paper

\shorttitle{Priesthood Roles}	% Short title used in the header

\authorsnames{Refaat Y. Yakoub}
\authorsaffiliations{
    {Research \& Development, Caterpillar Inc.}, 
    {The University of Illinois Research Park, Champaign, IL, USA},
    {St. Mina \& St. Pope Kyrillos VI COC, Champaign, IL, USA}
}

\course{SPIMS 2001: Research Methods}	% Course name
\professor{Dr. Nagwa Kalleny}	% Professor name
\duedate{March 2025}	% Due date

\authornote{This research is part of the course requirements for \textbf{SPIMS 2001: Research Methods}, taught by Dr. Nagwa Kalleny.  The work presented in this paper is a mere compilation of the biblical definition of priesthood and how the Fathers of the early Church understood and applied it throughout the generations.  The author refrained from  personal opinions and views to avoid any possible conflict of interest.  For correspondence, please contact the author at
\href{mailto:refaat.yakoub@gmail.com}{refaat.yakoub@gmail.com.}}

\abstract{
    This paper explores the essential role of the priesthood in the relationship between God and humanity through its theological, spiritual, and liturgical dimensions. It examines the priest’s role as a mediator between God and the faithful, focusing on evangelism, pastoral care, and sacramental administration. Drawing on biblical teachings, the writings of the early Church Fathers, and modern theological perspectives, this study highlights how the priesthood remains vital in contemporary times.  By analyzing these aspects, the study reaffirms the enduring necessity of priesthood in guiding the faithful toward salvation.
}

\keywords{priesthood, roles, evangelism, pastoral, sacramental.}

\begin{document}
\maketitle
\renewcommand{\baselinestretch}{1}
% set single space for the rest of the document
\renewcommand{\baselinestretch}{1}

\section{Introduction}
Is the priesthood necessary for humans to reach the Kingdom of God? This question frequently arises today, challenging the necessity of priesthood. This question can be addressed by examining the priesthood’s role in the God-human relationship, drawing from the Holy Bible and the traditions handed down from the Apostles to the early Church Fathers. The Holy Bible \citep{nelson_nkjv} defined priests as shepherds of the faithful (John 10:1-30), stewards of the sacraments (1 Corinthians 4:1, Titus 1:7), and representatives of Christ in the community (Matthew 16:19, John 20:22-23). The writings of the early Church Fathers, in the first four centuries after Christ (AD), laid a foundational understanding of the priesthood’s role in mediating the God-human relationship.  \citet{priesthood_chrysostom} reflected on the priest’s role as an ``icon of Christ'', entrusted with the sacred responsibility of leading the faithful to God through the sacraments. He emphasized the priest’s role in absolving sins (John 20:22-23), offering the Eucharist (Luke 22:19-20), and guiding souls, portraying the priesthood as indispensable to the spiritual health of the Church.  \citet{epistles_ignatius}, in his letters to churches and individuals, highlighted the priesthood as a continuation of Christ’s ministry through the Apostles. He emphasized the priest’s role in maintaining unity in the Body of Christ (Ephesians 4:11-13), making the relationship between God and humanity not only individualistic but communal. \citet{st_basil_letters} also commented on the priest’s important job in teaching people about God’s mysteries, especially the Eucharist, which is the most important way for humans to connect with God. (1 Corinthians 10:16-17). 

Despite historical divisions, the Orthodox and Catholic traditions have preserved this priestly role as described by the early Church Fathers.  \citet{orthodox_church_ware}, AKA Metropolitan Kallistos, explored the priesthood, from Orthodoxy, as both a sacramental and pastoral office. He explained that the priest acts as a visible sign of Christ’s presence within the Church, enabling the faithful to encounter God tangibly through the sacraments. \citet{eucharist_schmemann} added to this view by emphasizing the liturgical role of the priest in leading the community in worship and facilitating participation in the Kingdom of God. He described the priest as a bridge, through whom divine grace is channeled to humanity, especially in the Eucharist (Hebrews 9:11-14). \citet{priests_zacharias} added a pastoral dimension, describing the priest as a spiritual healer who guides individuals in their journey toward fellowship with God. His work shed light on how the priest’s role extended beyond liturgical functions to include nurturing the personal and communal aspects of the God-human relationship (James 5:14-16).  \cite{priesthood_shenouda} provided a Coptic Orthodox perspective, presenting the priesthood as a direct continuation of the Old Testament priestly office, fulfilled and transformed in Christ (Hebrews 7:23-28). He emphasized the role of the priest in reconciling humanity with God through sacraments and teaching (2 Corinthians 5:18-20). \citeauthor{priesthood_shenouda} also highlighted the priest as a servant and intercessor who embodies Christ’s love and humility, ensuring that the God-human relationship is rooted in both divine grace and human response.  \citet{early_church_akin} explored the Catholic's perspective also using the early Church’s understanding of the priesthood, emphasizing its sacrificial and sacramental nature. He discussed how the priesthood was seen as a continuation of the Old Testament priesthood, with the priest acting as a mediator between God and humanity (Hebrews 5:1-4). \citeauthor{early_church_akin} also highlighted, through the writing of the Early Church Fathers, the priest’s role in leading the faithful in worship, teaching, and pastoral care, underscoring the importance of the priesthood in the life of the Church.

While the Orthodox and Catholic Churches have preserved the priesthood as a sacred office, historical developments led to differing understandings of its role. Over time, concerns arose within the Catholic Church regarding the centralization of ecclesiastical authority and the increasing influence of clergy in secular matters (Matthew 20:25-28). These concerns were the catalyst of the Reformation movement \citep{reformation_lutzer}, which challenged aspects of the traditional priesthood.  \citet{reformation_lull_nelson} stated that reformers following Martin Luther's movement believed in the concept of the ‘priesthood of all believers’ (1 Peter 2:9), questioning the necessity of an intermediary priesthood. \citet{hebrews_wright} discussed Christ as the ultimate High Priest, arguing that His sacrifice fulfills and surpasses the Old Testament priesthood, thereby redefining its role (Hebrews 10:12-14). Similarly, \citet{reformation_macCulloch} explored how the Reformation led to a diminished sacramental role for the priesthood, shifting its primary function toward pastoral and evangelical roles.

Regardless of the differences, the fundamental goal of priesthood is to bring people into the Kingdom of God (Colossians 1:28). This paper discusses the role of priesthood in achieving this noble goal by exploring it from three, interconnected, dimensions; Evangelical, Pastoral, and Sacramental.  The paper doesn't discuss the administrative roles and ranks of the priesthood.  However, for the interested reader, \citet{epistles_ignatius} in his epistle to the Magnesians provided detailed definition to the offices of the Bishop, Presbyter, and the Deacon.  In the context of this paper, priesthood is considered as a general term for all ordained clergy, including bishops, priests, and deacons.

The following sections cover each dimension from the biblical foundation, early Church Fathers' understanding, and modern theologians' preaching.  The biblical references are taken from the New King James Version of the Orthodox Study Bible \citep{nelson_nkjv}.  For the sake of brevity, the biblical verses are not quoted in the text, however, references are given in parentheses following the common practice in theological writings of using the book name followed by the chapter number, colon, and the verse range number.  For example, (John 10:10-12) means verses 10 to 12 of Chapter 10 of the Book of John.  The full name of the book, without abbreviations, is given to avoid confusion for general readers of any background.  It is also thought to introduce the early Church Fathers who are referenced in this paper in the Appendix \citep{early_church_akin}.

\vskip 0.5cm
\paragraph{Aim}  \emph{What is the role of the priesthood in the God-human relationship?}

\section{Evangelism Role}\label{evangelism}
Understanding the priesthood begins with its evangelistic mission. From the time of Christ’s calling of the Apostles, the priesthood has been entrusted with spreading the Gospel and guiding souls toward salvation. The priest serves as both a preacher and a sacramental minister, ensuring that believers are continually nourished by the Word and sacraments. This role is explored through biblical foundations, early Church writings, and the sacramental dimension of evangelism.

\subsection{Biblical Foundation for Evangelism in the Priesthood}
The role of the priesthood in evangelism is rooted in Christ’s own calling of the Apostles to become “fishers of men” (Matthew 4:19). This divine assignment signifies the priest’s responsibility to bring people to God, guiding them toward salvation through teaching, preaching, and the sacraments. The priesthood continues the apostolic mission, ensuring the spread of the Gospel and the spiritual transformation of individuals and communities \citep{complete_work_chrysostom}.

The mission of evangelism was first entrusted to the Apostles, with Christ explicitly instructing them to “Go therefore and make disciples of all the nations, baptizing them in the name of the Father and of the Son and of the Holy Spirit” (Matthew 28:19). This command extends to the Church’s ordained clergy, who, as successors of the Apostles, fulfill Christ’s mission by teaching and baptizing believers into the faith \citep{complete_work_chrysostom}.

Additionally, St. Paul emphasized the necessity of preachers in spreading the Gospel, stating, “How then shall they call on Him in whom they have not believed? And how shall they believe in Him of whom they have not heard? And how shall they hear without a preacher?” (Romans 10:14). The priest serves as this preacher, ensuring that the message of salvation reaches both believers and non-believers alike.  This is a profound concept in evangelism explained by the Apostle Paul, who emphasized the importance of preaching the Gospel to all nations, as Christ commanded (Matthew 28:19-20).  The priest, as a successor of the Apostles, continues this mission of evangelism, proclaiming the Good News and inviting others to follow Christ.  St. Paul (Romans 15:16) described his own priestly ministry as a proclamation of the Gospel, sanctifying the Gentiles through the power of the Holy Spirit.  This is a clear example of how the priesthood is intimately connected to the mission of evangelism, bringing the Gospel to all people and inviting them to share in the life of Christ \citep{hgb-youssef_romans}.

\subsection{Evangelism in the Writings of the Church Fathers}

The early Church Fathers followed the footsteps of the Apostles modeling the priest’s role as an evangelist, emphasizing both the sacramental and pastoral duties involved in leading others to Christ. St. John Chrysostom highlighted the priest’s duty to preach the Word of God, stating that a priest must not only administer the sacraments but also “be ready at all times to teach and reprove, and to give instruction in the doctrines of the faith” \citep{,priesthood_chrysostom}. He compared the priest’s role to that of a shepherd who leads the lost back to the fold, mirroring Christ’s parable of the Good Shepherd (John 10:11-16).

St. Ignatius of Antioch also stressed the communal aspect of evangelism, asserting that priests, through their unity with the bishop, deacons, and the faithful, maintain the spiritual health of the Church (Ephesians 4:11-13). This emphasized the importance of the priest not just as an individual preacher but as part of a greater ecclesiastical mission \citep{grube_church_fathers}.

\subsection{The Sacramental Dimension of Evangelism}

Priests evangelize not only through preaching but also through administering the sacraments, which bring believers into full communion with God. The sacrament of baptism is central to evangelism, as it marks the believer’s birth in Christ. As Christ instructed Nicodemus, “Unless one is born of water and the Spirit, he cannot enter the kingdom of God” (John 3:5). The priest, acting in Christ’s behalf, facilitates this spiritual rebirth.  Furthermore, the Eucharist plays a vital role in strengthening and sustaining believers. \citet{st_basil_letters} taught that the Eucharist is the means by which Christians receive Christ Himself, deepening their faith and commitment to evangelism (1 Corinthians 10:16-17). By participating in the sacraments, believers are transformed and equipped to share the Gospel with others.  \citet{fanous_silent_patriarch} proved the power of the Eucharist in evangelism by the life of St. Pope Kyrillos VI, who was known for his deep faith and commitment to spreading the Gospel.  \citeauthor{fanous_silent_patriarch} showed how St. Pope Kyrillos VI’s life and ministry were rooted in the Eucharist, which sustained him in his mission of evangelism and pastoral care.  This is a clear example of how the sacraments can empower priests to fulfill their evangelistic calling despite the limitations and challenges they face in conducting the mission of evangelism as traditionally understood \citep{holmes_apostolic_fathers}.

\subsection{The Pastoral Dimension of Evangelism}

Beyond teaching and sacraments, evangelism also requires personal guidance and pastoral care.  This was evident in the exhortation of St. Paul to the presbyters of Ephesus when he summoned them before returning to Jerusalem \citep{paul_wright}.  Such teachings by examples have been preserved over the many generations of the Church. Pope Shenouda III, contemporary theologian and teacher, taught that the priest as a spiritual father who nurtures the faithful, helping them grow in their relationship with God \citep{priesthood_shenouda}. This aligns with St. Paul’s exhortation to Timothy: “Preach the word! Be ready in season and out of season. Convince, rebuke, exhort, with all long-suffering and teaching” (2 Timothy 4:2). The priest’s role in evangelism is thus not only about bringing new believers into the faith but also about strengthening and guiding those already within the Church who could suffer from complacency and hence may go astray \citep{bebawi_pope_kyrillos}.


\section{Pastoral Role}\label{pastoral}
While evangelism brings people into the faith, pastoral care ensures their spiritual growth and well-being. The priest’s responsibility does not end with conversion; rather, it extends into nurturing, guiding, and interceding for the faithful. As a spiritual shepherd, the priest provides personal counseling, teaches doctrine, and fosters unity within the Church. This section explores the pastoral dimensions of priesthood and how they serve the ongoing spiritual needs of believers.

\subsection{Biblical Foundation for the Pastoral Role}
The pastoral role of the priesthood is deeply embedded in biblical teachings and the tradition of the Church. As spiritual shepherds, priests provide guidance, care, and support to the faithful, ensuring their spiritual growth and well-being. This role encompasses teaching, counseling, and interceding for the community, reflecting Christ’s own ministry as the Good Shepherd (John 10:11-14).  The priest’s pastoral duties are rooted in Scripture, where spiritual leaders are entrusted with the care of God’s people \citep{complete_work_chrysostom}. In the Old Testament, priests and prophets acted as shepherds guiding Israel toward righteousness (Jeremiah 3:15). In the New Testament, Christ appointed His disciples to shepherd His flock, emphasizing love, humility, and service: ``Feed my sheep'' (John 21:17).

St. Paul provided clear pastoral instructions to the clergy, exhorting them to be ``blameless, as stewards of God'' (Titus 1:7) and to ``shepherd the church of God'' (Acts 20:28). He also emphasized patience, gentleness, and the ability to teach as essential qualities of a priest in his pastoral epistles \citep{hgb-youssef_pastoral_epsitles}.

\subsection{The Church Fathers on the Pastoral Role}

The early Church Fathers emphasized the priest’s role as a spiritual caregiver. \citet{priesthood_chrysostom} likened the priest to a physician, diagnosing and treating the illnesses of the soul. He warned that priests must exhibit great discernment, for their task is not merely administrative but deeply spiritual. Similarly, St. Gregory the Theologian \citep{early_church_akin} viewed pastoral care as a sacred art that required wisdom, compassion, and a deep understanding of human nature.

\citet{priesthood_shenouda} stressed the priest’s responsibility in guiding the faithful toward repentance and reconciliation with God, highlighting the importance of confession and spiritual direction (James 5:16). \citet{st_basil_letters} also underscored the necessity of personal holiness, arguing that a priest must first purify himself before leading others in the path of righteousness.

\subsection{Pastoral Care Through Teaching and Counseling}

A fundamental aspect of the pastoral role is teaching. Priests instruct the faithful in Scripture, theology, and Christian living, ensuring that they remain grounded in the truth (2 Timothy 3:16-17). St. Augustine \citep{early_church_akin} emphasized that priests must adapt their teaching to the needs of their audience, using simple language for the unlearned and deeper theological concepts for the mature.

Counseling is another critical aspect of pastoral care \citep{hgb-youssef_pastoral_epsitles}. Priests provide spiritual guidance in times of crisis, helping individuals navigate personal struggles, grief, and moral dilemmas. This pastoral responsibility aligns with St. Paul’s exhortation: ``Comfort each other and edify one another'' (1 Thessalonians 5:11).

\subsection{The Priest as an Intercessor}

The priest’s pastoral role also includes intercession—praying for and with the faithful. In both the Old and New Testaments, priests acted as intermediaries between God and His people (Exodus 28:29; Hebrews 5:1) as explained by \citet{hgb-youssef_hebrews}. The early Church continued this practice, as seen in liturgical prayers and intercessions for the sick, the troubled, and the departed \citep{holmes_apostolic_fathers}.

\citet{priests_zacharias} described the priest as a ``spiritual father'' emphasizing the necessity of constant prayer for the well-being of the community. The priest’s prayers, particularly in the Divine Liturgy, bring divine grace upon the faithful and sustain them in their spiritual journey.

\section{Sacramental Role}\label{sacramental}
Pastoral care is closely linked to the sacramental life of the Church. The sacraments, administered by the priest, are not just rituals but conduits of divine grace that sustain the faithful. Baptism initiates believers into the faith, the Eucharist nourishes them spiritually, and Confession restores their relationship with God. This section highlights the theological foundations of the sacramental role and its significance in the God-human relationship.

\subsection{Biblical Foundation for the Sacramental Role}

The priest’s sacramental role is central to the life of the Church. Through the sacraments, the priest acts as a conduit of God’s grace, facilitating the spiritual transformation of the faithful. The sacraments, particularly Baptism, Eucharist, and Confession, are not mere rituals but essential means of communion with God. The institution of the sacraments is deeply rooted in Scripture. Christ Himself established the sacraments and entrusted them to His disciples, who, in turn, passed them on through apostolic succession \citep{holmes_apostolic_fathers}. 

\begin{itemize}
    \item \textbf{Baptism}: Christ commanded His disciples, ``Go therefore and make disciples of all nations, baptizing them in the name of the Father and of the Son and of the Holy Spirit'' (Matthew 28:19). Baptism marks the believer’s entrance into the Church, and the priest, as a successor of the Apostles, administers this sacred rite.
    
    \item \textbf{Eucharist}: The Eucharist is the central sacrament of Christian life, instituted by Christ at the Last Supper: ``Take, eat; this is My body... Drink from it, all of you. For this is My blood of the new covenant'' (Matthew 26:26-28). The priest consecrates the bread and wine, making present the mystical reality of Christ’s sacrifice (1 Corinthians 10:16-17).

    \item \textbf{Confession and Absolution}: Christ granted His disciples the authority to forgive sins, saying, ``If you forgive the sins of any, they are forgiven them'' (John 20:23). The priest exercises this authority in the sacrament of Confession, guiding believers toward repentance and reconciliation with God (2 Corinthians 5:18-20).
\end{itemize}

\subsection{The Church Fathers on the Sacraments}

The early Church Fathers affirmed the centrality of the sacraments and the priest’s role in administering them. \cite{epistles_ignatius} warned against separation from the bishop and the priest, emphasizing that the Eucharist is valid only when offered by a legitimately ordained priest. 

St. Cyril of Jerusalem \citep{early_church_akin} described the sacraments as ``mysteries'' through which divine grace is imparted, ensuring the believer’s transformation into Christ-likeness. \citet{priesthood_chrysostom} highlighted the Eucharist’s significance in sustaining the faithful, referring to it as the ``medicine of immortality.''

\subsection{The Priest as a Mediator of Grace}

The priest does not act of his own authority but as an instrument of God’s grace. St. Gregory the Great \citep{early_church_akin} emphasized that the priest is a servant, not a master of the sacraments, administering them in obedience to Christ’s command.  The priest’s role in the sacraments is a continuation of Christ’s work, ensuring that divine grace reaches the faithful. \citet{priesthood_shenouda} stressed that the priest is not merely a functionary but a spiritual father who nourishes the Church through the sacraments.

\section{Contemporary Challenges and Opportunities}

In today’s world, the role of the priesthood faces new challenges and opportunities. Secularism, technological advancements, and changing societal perceptions impact how the priest fulfills his evangelistic, pastoral, and sacramental duties. While some argue that traditional priestly functions are becoming obsolete, the enduring presence of the priesthood remains crucial in addressing modern spiritual needs. This topic is much bigger than the scope of this paper, however, it is worth listing them for further research.

\clearpage
\section{Conclusions and Recommendations}

This paper has examined the priesthood’s role through evangelism, pastoral care, and sacramental ministry. These dimensions are interconnected, reinforcing the priest’s mission as a mediator between God and humanity. Evangelism spreads the Gospel, pastoral care nurtures the faithful, and sacraments channel divine grace.

Despite historical challenges, the priesthood remains indispensable in the Church. The early Church Fathers laid the foundation for its role, emphasizing its spiritual and sacramental necessity. While the Reformation questioned the need for intermediary priests, biblical and patristic teachings reaffirm that the priesthood continues Christ’s mission on Earth.

In contemporary society, priests must navigate secularism, digital ministry, and evolving pastoral needs. While these challenges pose difficulties, they also present opportunities for priests to engage with the faithful in new ways. Future research may explore how the priesthood can further integrate into modern contexts while upholding its traditional values.

Ultimately, the priesthood is a divine gift, serving as a bridge between humanity and God. By fulfilling their mission with faithfulness and humility, priests guide the faithful toward salvation and eternal life.

\vskip 0.5cm
\paragraph{Acknowledgments:} The author would like to thank the faculty and staff of St. Paul Institute for Ministry Studies for running this great spiritually-nurturing program.  Thanks are due to Dr. Nagwa Kalleny for her guidance and feedback on this paper. The author would like to thank Fr. James Mikhail of St. Paul COC, Chicago for reviewing the manuscript. Credits are due to Fr. Mina Beshara of St. Mina \& St. Pope Kyrillos VI COC, Champaign and Fr. John Beshara of St. Anthony COC, NJ for their ideas which helped in shaping the contents of this paper. Thanks are also due to Deacon Jason Fisher of The Good Shepherd Lutheran Church, Champaign for providing insights on the Lutheran perspective on the priesthood.

\appendix{}
\section{Who are the Early Church Fathers?}\label{early_fathers}
Throughout this paper, references have been made to the early Church Fathers, who played a significant role in shaping the understanding of the priesthood and its role in the Church.  Some are explicitly referenced in both the text and the bibliography.  Many others are referenced within a collection of sayings or writings. \citet{early_church_akin} provided an exhaustive list of the early Church Fathers who lived in the first four centuries of Christianity.  The following is a much shorter list of the early Church Fathers, with a brief bio, referenced explicitly or implicitly, in this paper:

\begin{itemize}
    \item \textbf{St. Ignatius of Antioch} (? - 110 AD): A disciple of St. John the Apostle, Ignatius was the third bishop of Antioch. He wrote seven epistles to various churches and individuals, emphasizing the importance of unity and the role of the bishop in maintaining the Church’s spiritual health.
    \item \textbf{St. John Chrysostom} (347-407 AD): Archbishop of Constantinople, Chrysostom was known for his eloquent preaching and pastoral care. He wrote extensively on the priesthood, emphasizing the priest’s role as an icon of Christ and a spiritual healer.
    \item \textbf{St. Basil the Great} (329-379 AD): Bishop of Caesarea, Basil was a prolific writer and theologian. He emphasized the importance of the Eucharist and the priest’s role in teaching and guiding the faithful.
    \item \textbf{St. Gregory the Theologian} (329-389 AD): Archbishop of Constantinople, Gregory was known for his theological writings and defense of the faith. He highlighted the priest’s pastoral and sacramental duties in leading the Church.
    \item \textbf{St. Cyril of Jerusalem} (313-386 AD): Bishop of Jerusalem, Cyril was a prominent theologian and catechist. He wrote extensively on the sacraments and the priest’s role in administering them.
    %\item \textbf{St. Gregory the Great} (540-604 AD): Pope of Rome, Gregory was known for his pastoral care and administrative reforms. He emphasized the priest’s role as a spiritual father and intercessor.
    \item \textbf{St. Augustine} (354-430 AD): Bishop of Hippo, Augustine was a prolific writer and theologian. He wrote extensively on the priesthood, emphasizing the sacramental and pastoral dimensions of the priest’s role.
\end{itemize}
    
The reason for limiting the choice to the first four centuries is obviously to focus on the early Church Fathers who laid the foundation of the Christian faith and the understanding of the priesthood before any splits which occurred after the  council of Chalcedon in year 451 AD \citep{early_church_akin}.  The list is not exhaustive, and many other Church Fathers have contributed to the understanding of the priesthood and its role in the Church, a topic should be investigated for further research.

\begin{thebibliography}{15}

\bibitem[Akin(2010)]{early_church_akin}Akin, J. (2010). \emph{The Fathers Know Best: Your Guide to the Teachings of the Early Church.} Catholic Answers Press.
    
\bibitem[Archimandrite Zacharias(2019)]{priests_zacharias}Archimandrite Zacharias. (2019). \emph{The Enlargement of the Heart.} Stavropegic Monastery of St. John the Baptist.

\bibitem[Bebawi(2022)]{bebawi_pope_kyrillos}Bebawi, G. (2022). \emph{My Holy Mentor: Spiritual Teachings of Pope Kyrillos VI.} Digital Kindle Publication.

\bibitem[Bishop Youssef(2013a)]{hgb-youssef_romans}Bishop Youssef. (2013a). \emph{Orthodox Christian Bible Commentary: Romans, Epistle of St. Paul.} St. Mary \& St. Moses Abbey, Sandia, TX.

\bibitem[Bishop Youssef(2013b)]{hgb-youssef_pastoral_epsitles}Bishop Youssef. (2013b). \emph{Orthodox Christian Bible Commentary: 1 \& 2 Timothy, Titus, Philemon, Epistles of St. Paul.} St. Mary \& St. Moses Abbey, Sandia, TX.

\bibitem[Bishop Youssef(2013c)]{hgb-youssef_hebrews}Bishop Youssef. (2013c). \emph{Orthodox Christian Bible Commentary: Hebrews, Epistle of St. Paul.} St. Mary \& St. Moses Abbey, Sandia, TX.


\bibitem[Fanous(2020)]{fanous_silent_patriarch}Fanous, D. (2020). \emph{A Silent Patriarch: Kyrillos VI (1902-1971): Life and Legacy.} St. Vladimirs Seminary Press.

\bibitem[Grube(2005)]{grube_church_fathers}Grube, G. W. (2005). \emph{What the Church Fathers Say About...} Light \& Life Publishing Company.

\bibitem[Holmes(2007)]{holmes_apostolic_fathers}Holmes, M. W. (2007). \emph{The Apostolic Fathers: Greek Texts and English Translations.} Baker Academic.

\bibitem[Lull and Nelson(2015)]{reformation_lull_nelson}Lull, T. F. and Nelson, D. R. (2015). \emph{Resilient Reformer, The Life and Thought of Martin Luther.} Fortress Press.

\bibitem[Lutzer(2017)]{reformation_lutzer}Lutzer, E. W. (2017). \emph{Rescuing the Gospel, the Story and Significance of the Reformation.} BakerBooks.

\bibitem[MacCulloch(2004)]{reformation_macCulloch}MacCulloch, D. (2004). \emph{The Reformation: A History.} Penguin Books.

\bibitem[NKJV(2008)]{nelson_nkjv}Nelosn, T. (2008). \emph{The Orthodox Study Bible.} St. Athanasius Academy of Orthodox Theology.

\bibitem[Pope Shenouda III(1989/1997)]{priesthood_shenouda}Pope Shenouda III. (1989/1997). \emph{The Priesthood, Translated by Amir Hanna.} COEPA. 

\bibitem[Schmemann(2010)]{eucharist_schmemann}Schmemann, A. (2010). \emph{For the Life of the World.} St. Vladimir's Seminary Press.
\bibitem[St. Basil the Great(2016)]{st_basil_letters}St. Basil the Great. (2016). \emph{The Letters.} Aeterna Press.
\bibitem[St. Ignatius of Antioch(2016)]{epistles_ignatius}St. Ignatius of Antioch (2024) \emph{The Epistles.} Kindle Edition.
\bibitem[St. John Chrysostom(1866)]{priesthood_chrysostom}St. John Chrysostom.  (1866). \emph{On the Priesthood, Translated by Harris Cowper.}  Williams and Norgate. Note: the date of the original manuscript is not available, the date of the translation is used here.
\bibitem[St. John Chrysostom(2016)]{complete_work_chrysostom}St. John Chrysostom.  (2016). \emph{The Complete Works of St. John Chrysostom.}  Digital Kindle Edition. Note: this is a collection of the complete works of St. John Chrysostom, the date of the collection and digital publication is used here. 
\bibitem[Ware(1997)]{orthodox_church_ware}Ware, T. (1997). \emph{The Orthodox Church, An Introduction to Eastern Christianity.} Aeterna Press.

\bibitem[Wright(2004)]{hebrews_wright}Wright, N.T. (2004). \emph{Hebrews for Everyone.} Westminster John Knox Press.


\bibitem[Wright(2018)]{paul_wright}Wright, N.T. (2018). \emph{Paul : A Biography.} Harper One.  Kindle Edition.


\end{thebibliography}

\end{document}  