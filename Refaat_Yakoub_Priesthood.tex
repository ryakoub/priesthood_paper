\documentclass[12pt, doc]{apa7}   	% use "apa7" for two columns format
\usepackage[authoryear,round]{natbib}


\title{The Roles of Priesthood in \dots TBC}
\shorttitle{Priesthood Roles}	% Short title used in the header

\authorsnames{Refaat Y. Yakoub}
\authorsaffiliations{
    {Research \& Development, Caterpillar Inc.}, 
    {The University of Illinois Research Park, Champaign, IL, USA},
    {St. Mina \& St. Pope Kyrillos VI COC, Champaign, IL, USA}
}

\course{SPIMS 2001: Research Methods}	% Course name
\professor{Dr. Nagwa Kalleny}	% Professor name
\duedate{March 2025}	% Due date

% \authornote{This research is part of the course requirements for SPIMS 2001: Research Methods, taught by Dr. Nagwa Kalleny.  The author would also like to ...  The author would like to thank Dr. Kalleny for her guidance and feedback on this paper.  The author is a member of the Coptic Orthodox Church and has a keen interest in theological studies.  For correspondence, please contact Refaat Y. Yakoub at
% \href{mailto:refaat.yakoub@gmail.com}{refaat.yakoub@gmail.com.}}

\abstract{
    This paper aims to shed some light on the role of the priesthood in the relationship between God and humanity by inspecting the theological, spiritual, and liturgical dimensions of this sacred office \dots
}

\keywords{priesthood, roles, evangelism, pastoral, sacramental, \dots}

\begin{document}
\maketitle

\section{Introduction}
The priesthood is a sacred office that has played a central role in the life of the Church since its inception. Priests are called to be mediators between God and humanity, guiding the faithful in their spiritual journey and administering the sacraments. The role of the priesthood is multifaceted, encompassing evangelism, pastoral care, and the administration of the sacraments. This paper explores the role of the priesthood in the God-human relationship from these three dimensions, drawing on biblical teachings, the writings of the early Church Fathers, and modern theological perspectives.
In Christianity, the priesthood is central to understanding the relationship between God and humanity. From the earliest days of the Church, priests have been seen as mediators, standing at the intersection of the divine and human realms. They act as shepherds of the faithful, stewards of the sacraments, and representatives of Christ in the community. The writings of the early Church Fathers, in the first four centuries of AD, laid a foundational understanding of the priesthood’s role in mediating the God-human relationship.  \citet{priesthood_chrysostom, complete_work_chrysostom} reflected on the priest’s role as an ``icon of Christ'', entrusted with the sacred responsibility of leading the faithful to God through the sacraments. He emphasizes the priest’s role in absolving sins (John 20:22-23), offering the Eucharist (Luke 22:19-20), and guiding souls, portraying the priesthood as indispensable to the spiritual health of the Church.  \citet{epistles_ignatius}, highlighted the priesthood as a continuation of Christ’s ministry through the Apostles. He emphasized the priest’s role in maintaining unity in the Body of Christ (Ephesians 4:11-13), making the relationship between God and humanity not only individualistic but communal. \citet{st_basil_letters} also talked about the priest’s important job in teaching people about God’s mysteries, especially the Eucharist, which is the most important way for humans to connect with God. (1 Corinthians 10:16-17). 

Despite differences and splits, this role, as described by the early Church Fathers, has been preserved in both the Orthodox and Catholic traditions.  \citet{orthodox_church_ware}, AKA Metropolitan Kallistos, explored the priesthood, from Orthodoxy, as both a sacramental and pastoral office. He explained that the priest acts as a visible sign of Christ’s presence within the Church, enabling the faithful to encounter God tangibly through the sacraments. \citet{eucharist_schmemann} added to this view by emphasizing the liturgical role of the priest in leading the community in worship and facilitating participation in the Kingdom of God. He described the priest as a bridge, through whom divine grace is channeled to humanity, especially in the Eucharist (Hebrews 9:11-14). \citet{priests_zacharias} added a pastoral dimension, describing the priest as a spiritual healer who guides individuals in their journey toward theosis (union with God). His work shed light on how the priest’s role extends beyond liturgical functions to include nurturing the personal and communal aspects of the God-human relationship (James 5:14-16).  \cite{priesthood_shenouda} provided a Coptic Orthodox perspective, presenting the priesthood as a direct continuation of the Old Testament priestly office, fulfilled and transformed in Christ (Hebrews 7:23-28). He emphasized the role of the priest in reconciling humanity with God through sacraments and teaching (2 Corinthians 5:18-20). \citeauthor{priesthood_shenouda} also highlighted the priest as a servant and intercessor who embodies Christ’s love and humility, ensuring that the God-human relationship is rooted in both divine grace and human response.  \citet{early_church_akin} explored the early Church’s understanding of the priesthood, emphasizing its sacrificial and sacramental nature. He discussed how the priesthood was seen as a continuation of the Old Testament priesthood, with the priest acting as a mediator between God and humanity (Hebrews 5:1-4). He also highlighted the priest’s role in leading the faithful in worship, teaching, and pastoral care, underscoring the importance of the priesthood in the life of the Church \citep{early_church_akin}.

While the Orthodox and Catholic Churches have preserved the priesthood as a sacred office, historical developments led to differing understandings of its role. Over time, concerns arose within the Catholic Church regarding the centralization of ecclesiastical authority and the increasing influence of clergy in secular matters (Matthew 20:25-28). These concerns contributed to the Reformation movement \citep{reformation_lutzer}, which challenged aspects of the traditional priesthood.  Reformers such as Martin Luther \citep{reformation_lull_nelson} emphasized the concept of the ‘priesthood of all believers’ (1 Peter 2:9), questioning the necessity of an intermediary priesthood. \citet{hebrews_wright} discussed Christ as the ultimate High Priest, arguing that His sacrifice fulfills and surpasses the Old Testament priesthood, thereby redefining its role (Hebrews 10:12-14). Similarly, \citet{reformation_macCulloch} explored how the Reformation led to a diminished sacramental role for the priesthood, shifting its primary function toward pastoral and evangelical services.

As it can be seen from the different views, the fundamental goal of priesthood is to bring people into the Kingdom of God (Colossians 1:28) This paper discusses the role of priesthood in achieving this noble goal by exploring it from three dimensions.  The following sections cover each dimension from the biblical foundation, early Church Fathers, and modern theologians.  The paper doesn't discuss the administrative roles of the priesthood.  However, for the interested reader, \citet{epistles_ignatius} in his epistle to the Magnesians provide detailed definition to the offices of the Bishop, Presbyter, and the Deacon.  

The biblical references are taken from the New King James Version of the Orthodox Study Bible \citep{nelson_nkjv}.  To avoid redundancy, the biblical references are cited in parentheses following the common practice in theological writings of using the book name followed by the chapter number, colon, and the verse range number.  For example, (John 10:10-12) means verses 10 to 12 of Chapter 10 of the Book of John.  For clarity, the full name of the book, without abbreviations, is given.  A list of the cited early Church Fathers is given in the Appendix as introduced by \citet{early_church_akin}.

\vskip 0.5cm
\paragraph{Aim}  \emph{What is the role of the priesthood in the God-human relationship?}



\section{Evangelism Role}\label{evangelism}
The role of the priesthood in evangelism is rooted in Christ’s own calling of the Apostles to become “fishers of men” (Matthew 4:19). This divine commission signifies the priest’s responsibility to bring people to God, guiding them toward salvation through teaching, preaching, and the sacraments. The priesthood continues the apostolic mission, ensuring the spread of the Gospel and the spiritual transformation of individuals and communities.

\subsection{Biblical Foundation for Evangelism in the Priesthood}

The mission of evangelism was first entrusted to the Apostles, with Christ explicitly instructing them to “Go therefore and make disciples of all the nations, baptizing them in the name of the Father and of the Son and of the Holy Spirit” (Matthew 28:19). This command extends to the Church’s ordained clergy, who, as successors of the Apostles, fulfill Christ’s mission by teaching and baptizing believers into the faith.

Additionally, St. Paul emphasized the necessity of preachers in spreading the Gospel, stating, “How then shall they call on Him in whom they have not believed? And how shall they believe in Him of whom they have not heard? And how shall they hear without a preacher?” (Romans 10:14). The priest serves as this preacher, ensuring that the message of salvation reaches both believers and non-believers alike.

\subsection{Evangelism in the Writings of the Church Fathers}

The early Church Fathers reinforced the priest’s role as an evangelist, emphasizing both the sacramental and pastoral duties involved in leading others to Christ. St. John Chrysostom highlighted the priest’s duty to preach the Word of God, stating that a priest must not only administer the sacraments but also “be ready at all times to teach and reprove, and to give instruction in the doctrines of the faith” \citep{,priesthood_chrysostom, complete_work_chrysostom}. He compared the priest’s role to that of a shepherd who leads the lost back to the fold, mirroring Christ’s parable of the Good Shepherd (John 10:11-16).

St. Ignatius of Antioch also stressed the communal aspect of evangelism, asserting that priests, through their unity with the bishop and the faithful, maintain the spiritual health of the Church (Ephesians 4:11-13). This underscored the importance of the priest not just as an individual preacher but as part of a greater ecclesiastical mission.

\subsection{The Sacramental Dimension of Evangelism}

Priests evangelize not only through preaching but also through administering the sacraments, which bring believers into full communion with God. The sacrament of baptism is central to evangelism, as it marks the believer’s entry into the Church. As Christ instructed Nicodemus, “Unless one is born of water and the Spirit, he cannot enter the kingdom of God” (John 3:5). The priest, acting in Christ’s behalf, facilitates this spiritual, in Christ, rebirth.

Furthermore, the Eucharist plays a vital role in strengthening and sustaining believers. \citet{st_basil_letters} taught that the Eucharist is the means by which Christians receive Christ Himself, deepening their faith and commitment to evangelism (1 Corinthians 10:16-17). By participating in the sacraments, believers are transformed and equipped to share the Gospel with others.  \citet{fanous_silent_patriarch} proved the power of the Eucharist in evangelism by the life of St. Pope Kyrillos VI, who was known for his deep faith and commitment to spreading the Gospel.  \citeauthor{fanous_silent_patriarch} showed how St. Pope Kyrillos VI’s life and ministry were rooted in the Eucharist, which sustained him in his mission of evangelism and pastoral care.  This is a clear example of how the sacraments can empower priests to fulfill their evangelistic calling despite the limitations and challenges they face in conducting the mission of evangelism as traditionally understood.

\subsection{The Pastoral Dimension of Evangelism}

Beyond teaching and sacraments, evangelism also requires personal guidance and pastoral care.  This was evident in the exhortation of St. Paul to the presbyters of Ephesus when he summoned them before returning to Jerusalem \citep{paul_wright}.  Such teachings by examples have been preserved over the many generations of the Church. Pope Shenouda III, contemporary theologian and teacher, taught that the priest as a spiritual father who nurtures the faithful, helping them grow in their relationship with God \citep{priesthood_shenouda}. This aligns with St. Paul’s exhortation to Timothy: “Preach the word! Be ready in season and out of season. Convince, rebuke, exhort, with all long-suffering and teaching” (2 Timothy 4:2). The priest’s role in evangelism is thus not only about bringing new believers into the faith but also about strengthening and guiding those already within the Church.

\subsection{Contemporary Challenges and Opportunities in Evangelism}

In today’s world, priests face new challenges in evangelism, including secularism, skepticism, and the rise of digital media. However, these also present opportunities for outreach. Modern Orthodox theologians, such as Timothy (Metropolitan Kallistos) Ware, emphasizes the importance of engaging with contemporary culture while remaining rooted in the traditions of the Church \citep{orthodox_church_ware}. Digital platforms, social media, and community outreach programs have become essential tools for priests in spreading the Gospel to a broader audience.


\section{Pastoral Role}\label{pastoral}

The pastoral role of the priesthood is deeply embedded in biblical teachings and the tradition of the Church. As spiritual shepherds, priests provide guidance, care, and support to the faithful, ensuring their spiritual growth and well-being. This role encompasses teaching, counseling, and interceding for the community, reflecting Christ’s own ministry as the Good Shepherd (John 10:11-14).

\subsection{Biblical Foundation for the Pastoral Role}

The priest’s pastoral duties are rooted in Scripture, where spiritual leaders are entrusted with the care of God’s people. In the Old Testament, priests and prophets acted as shepherds guiding Israel toward righteousness (Jeremiah 3:15). In the New Testament, Christ appointed His disciples to shepherd His flock, emphasizing love, humility, and service: ``Feed my sheep'' (John 21:17).

St. Paul provided clear pastoral instructions to the clergy, exhorting them to be ``blameless, as stewards of God'' (Titus 1:7) and to ``shepherd the church of God'' (Acts 20:28). He also emphasized patience, gentleness, and the ability to teach as essential qualities of a priest (2 Timothy 2:24-25).

\subsection{The Church Fathers on the Pastoral Role}

The early Church Fathers greatly emphasized the priest’s role as a spiritual caregiver. \citet{priesthood_chrysostom,complete_work_chrysostom} likened the priest to a physician, diagnosing and treating the ailments of the soul. He warned that priests must exhibit great discernment, for their task is not merely administrative but deeply spiritual. Similarly, St. Gregory the Theologian \citep{early_church_akin} viewed pastoral care as a sacred art that required wisdom, compassion, and a deep understanding of human nature.

\citet{priesthood_shenouda} stressed the priest’s responsibility in guiding the faithful toward repentance and reconciliation with God, highlighting the importance of confession and spiritual direction (James 5:16). \citet{st_basil_letters} also underscored the necessity of personal holiness, arguing that a priest must first purify himself before leading others in the path of righteousness.

\subsection{Pastoral Care Through Teaching and Counseling}

A fundamental aspect of the pastoral role is teaching. Priests instruct the faithful in Scripture, theology, and Christian living, ensuring that they remain grounded in the truth (2 Timothy 3:16-17). St. Augustine \citep{early_church_akin} emphasized that priests must adapt their teaching to the needs of their audience, using simple language for the unlearned and deeper theological discourse for the mature.

Counseling is another critical aspect of pastoral care. Priests provide spiritual guidance in times of crisis, helping individuals navigate personal struggles, grief, and moral dilemmas. This pastoral responsibility aligns with St. Paul’s exhortation: ``Comfort each other and edify one another'' (1 Thessalonians 5:11).

\subsection{The Priest as an Intercessor}

The priest’s pastoral role also includes intercession—praying for and with the faithful. In both the Old and New Testaments, priests acted as intermediaries between God and His people (Exodus 28:29; Hebrews 5:1). The early Church continued this practice, as seen in liturgical prayers and intercessions for the sick, the troubled, and the departed.

\citet{priests_zacharias} described the priest as a ``spiritual father'' emphasizing the necessity of constant prayer for the well-being of the community. The priest’s prayers, particularly in the Divine Liturgy, bring divine grace upon the faithful and sustain them in their spiritual journey.

\subsection{Contemporary Challenges in Pastoral Ministry}

Modern priests face new pastoral challenges, including the rise of secularism, mental health crises, and the digital age’s impact on faith communities. While these challenges can hinder traditional pastoral work, they also present opportunities for priests to adapt their approach. The use of digital platforms for pastoral counseling, online teaching, and community engagement has become essential in contemporary ministry \citep{orthodox_church_ware}. 

Despite these challenges, the priesthood remains a vital pastoral presence, guiding the faithful through personal interaction, sacramental ministry, and spiritual counsel.

\section{Sacramental Role}\label{sacramental}

The priest’s sacramental role is central to the life of the Church. Through the sacraments, the priest acts as a conduit of God’s grace, facilitating the spiritual transformation of the faithful. The sacraments, particularly Baptism, Eucharist, and Confession, are not mere rituals but essential means of communion with God.

\subsection{Biblical Foundation for the Sacramental Role}

The institution of the sacraments is deeply rooted in Scripture. Christ Himself established the sacraments and entrusted them to His disciples, who, in turn, passed them on through apostolic succession. 

\begin{itemize}
    \item \textbf{Baptism}: Christ commanded His disciples, ``Go therefore and make disciples of all nations, baptizing them in the name of the Father and of the Son and of the Holy Spirit'' (Matthew 28:19). Baptism marks the believer’s entrance into the Church, and the priest, as a successor of the Apostles, administers this sacred rite.
    
    \item \textbf{Eucharist}: The Eucharist is the central sacrament of Christian life, instituted by Christ at the Last Supper: ``Take, eat; this is My body... Drink from it, all of you. For this is My blood of the new covenant'' (Matthew 26:26-28). The priest consecrates the bread and wine, making present the mystical reality of Christ’s sacrifice (1 Corinthians 10:16-17).

    \item \textbf{Confession and Absolution}: Christ granted His disciples the authority to forgive sins, saying, ``If you forgive the sins of any, they are forgiven them'' (John 20:23). The priest exercises this authority in the sacrament of Confession, guiding believers toward repentance and reconciliation with God.
\end{itemize}

\subsection{The Church Fathers on the Sacraments}

The early Church Fathers affirmed the centrality of the sacraments and the priest’s role in administering them. \cite{epistles_ignatius} warned against separation from the bishop and the priest, emphasizing that the Eucharist is valid only when offered by a duly ordained priest. 

St. Cyril of Jerusalem \citep{early_church_akin} described the sacraments as ``mysteries'' through which divine grace is imparted, ensuring the believer’s transformation into Christlikeness. \citet{priesthood_chrysostom,complete_work_chrysostom} highlighted the Eucharist’s significance in sustaining the faithful, referring to it as the ``medicine of immortality.''

\subsection{The Priest as a Mediator of Grace}

The priest does not act of his own authority but as an instrument of God’s grace. St. Gregory the Great \citep{early_church_akin} emphasized that the priest is a servant, not a master of the sacraments, administering them in obedience to Christ’s command.

The priest’s role in the sacraments is a continuation of Christ’s work, ensuring that divine grace reaches the faithful. \citet{priesthood_shenouda} stressed that the priest is not merely a functionary but a spiritual father who nourishes the Church through the sacraments.

\subsection{Contemporary Challenges in the Sacramental Role}

The modern world poses challenges to the sacramental life, including declining participation in church services and a diminished understanding of the sacraments’ significance. However, priests can respond to these challenges by educating the faithful on the meaning of the sacraments, emphasizing their transformative power, and ensuring that sacramental life remains central to Christian practice \citep{eucharist_schmemann}.

\section{Effectiveness of the Priesthood in the God-Human Relationship}
The discussion of the priesthood’s role in the God-human relationship from the Evangelical, Pastoral, and Sacramental dimensions reveals the priest’s indispensable function in leading souls to God. However, the effectiveness of the priesthood in fulfilling this role depends on two key factors: the priest’s personal holiness and the faithful’s receptivity to God’s grace.  In the Holy Bible, Jesus Christ clearly defined the good shepherd as the one who lays down his life for the sheep (John 10:11).  The priest, as a shepherd, must emulate Christ’s sacrificial love, humility, and compassion in guiding the faithful.  The priest’s personal holiness is essential in reflecting Christ’s image to the community, inspiring trust, and fostering spiritual growth.  The faithful, in turn, must be open to receiving God’s grace through the priest, participating actively in the sacraments, and responding to the priest’s pastoral care.  The effectiveness of the priesthood in the God-human relationship is thus a dynamic interplay between the priest’s ministry and the faithful’s response to God’s grace.  The aposltes and the early Church Fathers have set the example of how the priesthood should be conducted in both by teaching and by example. 

As for the responsibility of the congregation in making this relationship effective, the faithful must be willing to receive the priest’s guidance, participate in the sacraments, and respond to the priest’s pastoral care.  The faithful must also pray for their priests, support them in their ministry, and hold them accountable to their sacred duties.  The effectiveness of the priesthood in the God-human relationship is thus a shared responsibility between the priest and the faithful, each contributing to the spiritual health and growth of the Church.  The faithful must also be willing to receive the priest’s guidance, participate in the sacraments, and respond to the priest’s pastoral care.  The faithful must also pray for their priests, support them in their ministry, and hold them accountable to their sacred duties.  The effectiveness of the priesthood in the God-human relationship is thus a shared responsibility between the priest and the faithful, each contributing to the spiritual health and growth of the Church. \dots TBC

\section{Conclusions and Recommendations}

This paper has explored the role of the priesthood from three perspectives: the Evangelical, Pastoral, and Sacramental dimensions. The priest is not merely a preacher or administrator but a mediator of grace, a spiritual father, and a shepherd to the faithful. 

Despite historical challenges and contemporary struggles, the priesthood continues to play a vital role in leading souls to God. Future research may explore how traditional pastoral methods can be adapted to modern challenges. \dots TBC


\appendix{}
\section{Who are the Early Church Fathers?}\label{early_fathers}
Throughout this paper, references have been made to the early Church Fathers, who played a significant role in shaping the understanding of the priesthood and its role in the Church.  Some are explicitly referenced in both the text and the bibliography.  Many others are referenced within a collection of sayings or writings. \citet{early_church_akin} provided an exhaustive list of the early Church Fathers who lived in the first four centuries of Christianity.  The following is a much shorter list of the early Church Fathers, with a brief bio, referenced explicitly or implicitly, in this paper:

\begin{itemize}
    \item \textbf{St. Ignatius of Antioch} (? - 110 AD): A disciple of St. John the Apostle, Ignatius was the third bishop of Antioch. He wrote seven epistles to various churches and individuals, emphasizing the importance of unity and the role of the bishop in maintaining the Church’s spiritual health.
    \item \textbf{St. John Chrysostom} (347-407 AD): Archbishop of Constantinople, Chrysostom was known for his eloquent preaching and pastoral care. He wrote extensively on the priesthood, emphasizing the priest’s role as an icon of Christ and a spiritual healer.
    \item \textbf{St. Basil the Great} (329-379 AD): Bishop of Caesarea, Basil was a prolific writer and theologian. He emphasized the importance of the Eucharist and the priest’s role in teaching and guiding the faithful.
    \item \textbf{St. Gregory the Theologian} (329-389 AD): Archbishop of Constantinople, Gregory was known for his theological writings and defense of the faith. He highlighted the priest’s pastoral and sacramental duties in leading the Church.
    \item \textbf{St. Cyril of Jerusalem} (313-386 AD): Bishop of Jerusalem, Cyril was a prominent theologian and catechist. He wrote extensively on the sacraments and the priest’s role in administering them.
    %\item \textbf{St. Gregory the Great} (540-604 AD): Pope of Rome, Gregory was known for his pastoral care and administrative reforms. He emphasized the priest’s role as a spiritual father and intercessor.
    \item \textbf{St. Augustine} (354-430 AD): Bishop of Hippo, Augustine was a prolific writer and theologian. He wrote extensively on the priesthood, emphasizing the sacramental and pastoral dimensions of the priest’s role.
\end{itemize}
    

The reason for limiting the choice to the first four centuries is obviously to focus on the early Church Fathers who laid the foundation of the Christian faith and the understanding of the priesthood before any splits which occurred after the  council of Chalcedon in year 451 AD \citep{early_church_akin}.  The list is not exhaustive, and many other Church Fathers have contributed to the understanding of the priesthood and its role in the Church.  The list is provided here for reference and further study. \dots TBC

\begin{thebibliography}{15}

\bibitem[Akin(2010)]{early_church_akin}Akin, J. (2010) \emph{The Fathers Know Best: Your Guide to the Teachings of the Early Church.} Catholic Answers Press.
    
\bibitem[Archimandrite Zacharias(2019)]{priests_zacharias}Archimandrite Zacharias (2019) \emph{The Enlargement of the Heart.} Stavropegic Monastery of St. John the Baptist.

\bibitem[Bebawi (2022)]{bebawi_pope_kyrillos}Bebawi, G. (2022) \emph{My Holy Mentor: Spiritual Teachings of Pope Kyrillos VI.} Kindle Edition.

\bibitem[Fanous(2020)]{fanous_silent_patriarch}Fanous, D. (2020) \emph{A Silent Patriarch: Kyrillos VI (1902-1971): Life and Legacy.} St. Vladimirs Seminary Press.

\bibitem[Lull and Nelson(2015)]{reformation_lull_nelson}Lull, T. F. and Nelson, D. R. (2015) \emph{Resilient Reformer, The Life and Thought of Martin Luther.} Fortress Press.

\bibitem[Lutzer(2017)]{reformation_lutzer}Lutzer, E. W. (2017) \emph{Rescuing the Gospel, the Story and Significance of the Reformation.} BakerBooks.

\bibitem[MacCulloch(2004)]{reformation_macCulloch}MacCulloch, D. (2004) \emph{The Reformation: A History.} Penguin Books.

\bibitem[NKJV(2008)]{nelson_nkjv}Nelosn, T (2008) \emph{The Orthodox Study Bible.} St. Athanasius Acadamy of Orthodox Theology.

\bibitem[Pope Shenouda III(1997)]{priesthood_shenouda}Pope Shenouda III (1997) \emph{The Priesthood, Translated by Amir Hanna.} COEPA. 

\bibitem[Schmemann(2010)]{eucharist_schmemann}Schmemann, A. (2010) \emph{For the Life of the World.} St. Vladimir's Seminary Press.
\bibitem[St. Basil the Great(2016)]{st_basil_letters}St. Basil the Great (2016) \emph{The Letters.} Aeterna Press.
\bibitem[St. Ignatius of Antioch(2016)]{epistles_ignatius}St. Ignatius of Antioch (2024) \emph{The Epistles.} Kindle Edition.
\bibitem[St. John Chrysostom(1866)]{priesthood_chrysostom}St. John Chrysostom  (1866) \emph{On the Priesthood, Translated by Harris Cowper.}  Williams and Norgate. 
\bibitem[St. John Chrysostom(2006)]{complete_work_chrysostom}St. John Chrysostom  (2016) \emph{The Complete Works of St. John Chrysostom.}  Kindle Edition. 
\bibitem[Ware(1997)]{orthodox_church_ware}Ware, T. (1997) \emph{The Orthodox Church, An Introduction to Eastern Christianity.} Aeterna Press.

\bibitem[Wright(2004)]{hebrews_wright}Wright, N.T. (2004) \emph{Hebrews for Everyone.} Westminster John Knox Press.


\bibitem[Wright(2018)]{paul_wright}Wright, N.T. (2018) \emph{Paul : A Biography.} Harper One.  Kindle Edition.


\end{thebibliography}

\end{document}  